\documentclass{article}
\usepackage[utf8]{inputenc}
\usepackage[spanish]{babel}
\usepackage{listings}
\usepackage{graphicx}
\graphicspath{ {Images/} }
\usepackage{cite}

\begin{document}

\begin{titlepage}
    \begin{center}
        \vspace*{1cm}
            
        \Huge
        \textbf{Proyecto Final}
            
        \vspace{0.5cm}
        \LARGE
        Informe Escrito
            
        \vspace{1.5cm}
            
        \textbf{Julian Taborda Ramirez}
        
        \vspace{0.5cm}
        
        \textbf{Samuel Ruiz Vargas}
            
        \vfill
            
        \vspace{0.8cm}
            
        \Large
        Informática II\\
        Universidad de Antioquia\\
        Medellín\\
        Octubre de 2021
            
    \end{center}
\end{titlepage}

\tableofcontents
\vspace*{1.2cm}

\newpage

\section{Planificación}
        Iniciamos identificando nuestras tareas, una vez hecho esto comenzamos a dividir nuestro trabajo de manera que, pudiéramos trabajar eficientemente, para ello cada uno escogió las tareas con la que se sintiera más cómodo. Posteriormente realizamos un cronograma basándonos en la importancia y el tiempo requerido para cada tarea. El plan fue seguir el cronograma lo más fiel posible sin embargo siempre estuvo sujeto a cambios. 

\section{Avances Regulares}
    \subsection{9/10/2021}
        Este día fue la apertura del telón para comenzar nuestro proyecto final, aquí nos encargamos de lo relacionado a la planificación del proyecto así como a iniciar este informe. Además de esto creamos una pagina en Notion para mantenernos al tanto de las actividades y siempre estar organizados (Link al Notion en el README del repositorio).
    
\section{Modelamiento de las Clases}
    \subsection{}
        
        
\section{Estructura del Código}
    \subsection{}
    
    
\section{Problemas Presentados}
    \subsection{}

\end{document}
